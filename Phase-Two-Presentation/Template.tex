\documentclass[xcolor={table}, aspectratio=169]{beamer}
%
% Choose how your presentation looks.
%
% For more themes, color themes and font themes, see:
% http://deic.uab.es/~iblanes/beamer_gallery/index_by_theme.html
%
\defbeamertemplate{footline}{my frame number}
{
    \hfill
    \usebeamercolor[fg]{page number in head/foot}
    \usebeamerfont{page number in head/foot}
    \raisebox{0.25cm}[0pt][0pt]{
        \insertframenumber\kern1em}
}
\mode<presentation>
{
    \usetheme{default}
    \usecolortheme{seagull}
    \usefonttheme{default}
    \setbeamertemplate{caption}[numbered]
    \setbeamertemplate{footline}[my frame number]
    \setbeamertemplate{navigation symbols}{}
    \addtobeamertemplate{frametitle}{\vskip+0.5ex}{}
}
% \AtBeginSection[] {
%     \begin{frame}<beamer>
%     \frametitle{Outline}
%     \tableofcontents[currentsection]
%     \end{frame}
% }
% \AtBeginSubsection[] {
%     \begin{frame}<beamer>
%     \frametitle{Outline}
%     \tableofcontents[currentsection, currentsubsection]
%     \end{frame}
% }

\usepackage[english]{babel}
\usepackage[export]{adjustbox}
\usepackage[permil]{overpic}
\usepackage{threeparttable}
\usepackage{graphicx}
\usepackage{subcaption}
\usepackage{setspace}
\usepackage{ulem}

\usepackage{subcaption}
\usepackage{psfrag}
%\usepackage{natbib}
\usepackage[autocite=superscript,backend=bibtex, style=authoryear]{biblatex}
\addbibresource{refs.bib}

\usepackage{multirow}
\usepackage{hyperref}
\usepackage{graphicx}
\usepackage{color}
% \usepackage{cleveref}
\usepackage[font=scriptsize]{caption}
\usepackage{booktabs}
\usepackage{amsmath}

\usepackage{siunitx}

\sisetup{output-exponent-marker=\ensuremath{\mathrm{e}}}



%\usepackage{fontspec}

\graphicspath{{stevens/}{figures/}}

\makeatletter
\def\beamer@framenotesbegin{% at beginning of slide
     \usebeamercolor[fg]{normal text}
      \gdef\beamer@noteitems{}%
      \gdef\beamer@notes{}%
}
\makeatother

\usebackgroundtemplate
{
    \begin{tabular}{@{}c@{}}
        \includegraphics[width=\paperwidth]{presentation_top.png} \\
        \rule{0pt}{0.74\paperheight} \\
        \includegraphics[width=\paperwidth]{presentation_bottom.png}
    \end{tabular}
}

\vspace{1cm}

\title[Lit Review]
      {Phase Two Presentation: RL vs. Naive Diversification\\
      \large{Robustness and Frictions Across Market Regimes}}

\author[Author] % (optional, use only with lots of authors)
{Andre Sealy}
% - Use the \inst{?} command only if the authors have different
%   affiliation.

\institute[Stevens Institute] % (optional, but mostly needed)
{
	Advisor: Jingrui (Victoria) Li\\
  Financial Engineering 800\\
  Stevens Institute of Technology\\
%System description details at \cite{w2020shift}}
}
\date{\\
{\footnotesize November 11th, 2025}}

\subject{Talk at Venue}


%\author[Ionut Florescu]{\texorpdfstring{\footnotesize Ionu\c{t} Florescu \\
%       \vspace*{0.5\baselineskip}
%     {\footnotesize Committee: \\
%      Dr. George Calhoun, Dr. Dragoș Bozdog, \\
%      Dr. Emmanuel Hatzakis, Dr. Rupak Chatterjee}}{Ionu\c{t} Florescu}}
% \institute{\footnotesize Financial Engineering}
% \date{{\footnotesize September 12, 2020}}

\begin{document}

{
\setbeamertemplate{footline}{}
\usebackgroundtemplate
{
	\begin{tabular}{@{}c@{}}
		\begin{overpic}[width=\paperwidth]{title_top.png}
			\put(25,-35){\includegraphics[height=0.15\paperheight]{title_logo.png}}
		\end{overpic} \\
		\rule{0pt}{0.74\paperheight}                     \\
		\includegraphics[width=\paperwidth]{title_bottom.png}
	\end{tabular}
}

\begin{frame}

	\titlepage

\end{frame}
}

\begin{frame}{Overview}

	\begin{itemize}
		\item Addressing computational issues in the trajectory state extraction from the DRL
		      \newline
		\item Incorporating the regimes from the Deep Learning in Asset Pricing Paper (Chen et al.)
	\end{itemize}

\end{frame}

\begin{frame}{Computational Issues}
	Running reinforcement learning (RL) algorithms with decision transformers (DT) across many different states introduces several computational challenges, such as,
	\begin{itemize}
		\item \textbf{High Memory Requirements:} Significant amounts of memory are required for storing and processing trajectories.
		\item \textbf{Data Inefficiencies:} Many different states demand more samples to cover the space adequately, compounding dataset size and training time.
		\item \textbf{Sequence Alignment and Temporal Credit Assignment:} The more states the DT has, aligning and credit outcomes to earlier actions becomes computationally more difficult.
		\item \textbf{Model Size and Training Cost:} DT may need larger transformer models, which require more computational resources, longer training times, and powerful hardware.
	\end{itemize}
\end{frame}

\begin{frame}{High Performance Computing}
	High-performance computing (HPC) can significantly accelerate research on RL for portfolio optimization, especially when using computationally demanding models like decision transformers in complex regimes.
	\begin{itemize}
		\item \textbf{Parallelization and Speed:} HPC environments allows computational to be distributed across many CPUs and GPUs, drastically reducing the time for training RL models and grid searches over hyperparameters.
		\item \textbf{Larger Models and Datasets:} We can train deeper transformer architectures and process longer trajectories or higher-dimensional state spaces, overcoming the memory and compute limits of a regular PC.
		\item \textbf{Handling Frictions and Constraints:} Simulating realistic trading frictions (transaction costs, slippage, etc.) requires substantial computation and state tracking.
	\end{itemize}
\end{frame}

\begin{frame}{JARVIS}
	\begin{columns}
		\column{0.6\textwidth}
		Stevens Institute of Technology's HPC cluster JARVIS is a state-of-the-art computing resource designed to support advanced research across the university.
		\begin{itemize}
			\item \textbf{Compute Resources:} JARVIS consists 55 codes, providing 3,168 CPU cores and 32 GPUs, including 8 advanced Nvidia L40s GPUs.
			\item \textbf{Memory:} The cluster has 14 TBs of memory, which enables it to handle complex, memory-intensive workloads.
			\item \textbf{Storage:} It includes 1.2 petabytes (PB) of storage, supporting larged dataset and model checkpoint management.
		\end{itemize}
		\column{0.5\textwidth}
		\begin{figure}
			\includegraphics[width=1.\textwidth]{./pics/jarvis.jpeg}
			\caption{Not actually Steven's JARVIS}
		\end{figure}
	\end{columns}
\end{frame}

\begin{frame}{HPC Motivations}
	\begin{itemize}
		\item Utilizing S\&
	\end{itemize}

\end{frame}

\end{document}
